\documentclass[a4paper,11pt]{article}
\usepackage[utf8]{inputenc}
\usepackage[T1]{fontenc}
\usepackage[a4paper]{geometry}
\usepackage{comment}
\usepackage{tocloft}
\usepackage{longtable}
\usepackage{booktabs}
\usepackage{array}
\usepackage{color}
\usepackage{parskip}
\usepackage{fancyvrb}
\usepackage[hidelinks]{hyperref}
\hypersetup{
hyperfootnotes=false,
breaklinks=true,
colorlinks=false
}

\newcommand{\name}[1]{\textit{#1}}
\newcommand{\f}[1]{\texttt{#1}}
\newcommand{\m}[1]{\mathit{#1}}
\newcommand{\var}[1]{\textnormal{\textit{#1}}}
\newcommand{\lit}[1]{\textnormal{\textit{#1}}}

\newcommand{\propsig}[2]
{\noindent \f{#1} : #2}

\newcommand{\propdesc}[1]
{\par \hspace*{\fill}\begin{minipage}{0.9\textwidth}#1\end{minipage}\par\smallskip}

\newcolumntype{L}[1]
{>{\raggedright\let\newline\\\arraybackslash\hspace{0pt}}p{#1}}
\newcolumntype{C}[1]
{>{\centering\let\newline\\\arraybackslash\hspace{0pt}}p{#1}}
\newcolumntype{R}[1]
{>{\raggedleft\let\newline\\\arraybackslash\hspace{0pt}}p{#1}}
\newcolumntype{M}[1]
{>{\raggedleft\let\newline\\\arraybackslash\hspace{0pt}$}p{#1}<$}
\newcolumntype{S}{@{\hspace{4pt}}}

%%%%%%%%%%%%%%%%%%%%%%%%%%%%%%%%%%%%%%%%%%%%%%%%%%%%%%%%%%%%%%%%%%%%%%%%%%%%%%

% \newcommand{\supterm}{\mathrel{\rhd}}
% \newcommand{\suptermq}{\mathrel{\unrhd}}
% \newcommand{\notsuptermq}{\not \mathrel{\unrhd}}
% \newcommand\subsumedBy{\mathrel{\ooalign{$\geq$\cr
%       \hidewidth\raise.225ex\hbox{$\cdot\mkern7.0mu$}\cr}}}
% \newcommand\subsumes{\mathrel{\ooalign{$\leq$\cr
%       \hidewidth\raise.225ex\hbox{$\cdot\mkern2.0mu$}\cr}}}
% \newcommand\strictlySubsumes{\mathrel{\ooalign{$<$\cr
%       \hidewidth\raise.0ex\hbox{$\cdot\mkern2.0mu$}\cr}}}
% \newcommand\strictlySubsumedBy{\mathrel{\ooalign{$>$\cr
%        \hidewidth\raise.0ex\hbox{$\cdot\mkern7.0mu$}\cr}}}
% \newcommand\variant{\mathrel{\ooalign{$=$\cr
%       \hidewidth\raise.7ex\hbox{$\cdot\mkern4.5mu$}\cr}}}

%%%%%%%%%%%%%%%%%%%%%%%%%%%%%%%%%%%%%%%%%%%%%%%%%%%%%%%%%%%%%%%%%%%%%%%%%%%%%%

\title{<PROJECT-NAME> Training Data Format}
\author{WB, JO, MR, CW, ZZ}

\date{Draft -- \today}

\begin{document}
\maketitle


\section{General Remarks}

Our concrete data representation is readable with SWI-Prolog.

We consider entities of three types: \f{problem}, \f{proof} and \f{lemma}.

Data are stored in triple facts
\begin{center}
\f{spo}(\var{Subject}, \var{Property}, \var{Object}).
\end{center}

A triple with the \f{type} property is present for all entities that appear as
\var{Subject} in some triple.

\var{Object} can be an identifier of another entity appearing as \var{Subject}
in some triple or can be literal value, that is, a term as specified in
Sect.~\ref{sec-literal-types}. \var{Property} can be an atom or a compound
term that is ground. Examples for the latter are properties \f{axiom(1)} and
\f{axiom(2)}.

Entity identifiers are atoms. Their concrete form depends on the entity type.

% TODO (maybe): indicate which properties are expected to be present and which
% are optional. TODO (maybe): outline how far this is tied to CD and how it
% can be generalized.

\subsection{Terminology: Size Measures}
\begin{longtable}{lL{28em}}
  \emph{Tree size} & The tree size of a tree or term is the number of inner
  nodes, or the number of occurrences of function symbols with arity $\geq 1$.
  The tree size of a multiset of trees or terms is the sum of the tree sizes
  of its members.\\
 \emph{Height} & The height of a tree of term is the number
  of edges of the longest downward path from the root to a leaf. The height of
  a set of trees or terms is the maximum of the height of its members.\\
  \emph{Compacted size} & The compacted size of a tree or term is the number of
  inner nodes of the minimal DAG representing the tree, or the number of
  distinct compound subterms. The compacted size of a set of trees or terms is
  number of inner nodes of the minimal DAG representing the set, or the number
  of distinct compound subterms of the members of the set.
\end{longtable}

We also talk about applying these size measures to a \emph{list}, as a data
object, which is then taken to represent the multiset or set, respectively, of
its members.

% Height and compacted size could also be defined on multisets, but for any
% given multiset the value would be the same as the value for the set of its
% members.

\subsection{Literal Types}
\label{sec-literal-types}

As types of literal values of we consider the following.

\begin{longtable}{lL{28em}}
  \lit{Term} & An arbitrary Prolog term.\\
  \lit{Atom} & A Prolog atom.\\
  \lit{NatNum} & A natural number.\\
  \lit{NormalizedValue} & A number between 0 and 1.\\
  \lit{FormulaTerm} & A term that is understood as representing an atomic
  first-order formula. Free variables are represented by Prolog variables. For
  CD problems, the unary wrapper predicate is omitted and \lit{FormulaTerm} is
  a propositional formula of the problem's ``object logic'' with function
  symbols \f{i}/2, \f{n}/1 and \f{falsehood}/0. E.g., a clausal axiom
  \f{is\_a\_theorem}(\f{implies}(X,\f{implies}(Y,X))) would
  be represented by the \lit{FormulaTerm}
  \f{i}(X,\f{i}(Y,X)).\\
  \lit{FormulaTermList} & A list of objects of type \lit{FormulaTerm}.\\
  \lit{DTerm} & A term, called D-term, that is understood as representing the
  proof structure of a CD proof. The term is formed from $\f{d}/2$ as function
  symbol, objects of type
  \lit{NatNum}, and variables.\\
  \lit{DCTerm} & A compacted representation of a \lit{DTerm}. For a given
  \lit{DTerm}~\lit{D} the \lit{DCTerm} is \lit{A}\f{-}\lit{B} as obtained with
  \texttt{term\_factorized}(\lit{D}, \lit{A}, \lit{B}). See
  Sect.~\ref{sec-aux-utils} for a conversion predicate in the opposite
    direction.\\
  \lit{KeyValueList} & A list of \lit{Key}\f{=}\lit{Value} pairs.\\
\end{longtable}

%%%%%%%%%%%%%%%%%%%%%%%%%%%%%%%%%%%%%%%%%%%%%%%%%%%%%%%%%%%%%%%%%%%%%%%%%%%%%%
%%%%%%%%%%%%%%%%%%%%%%%%%%%%%%%%%%%%%%%%%%%%%%%%%%%%%%%%%%%%%%%%%%%%%%%%%%%%%%
\section{Entity Types and Associated Properties}

%%%%%%%%%%%%%%%%%%%%%%%%%%%%%%%%%%%%%%%%%%%%%%%%%%%%%%%%%%%%%%%%%%%%%%%%%%%%%%
\subsection{Type \f{problem}}

Represents a proving problem, that is, a goal and a finite set of axioms.

The identifier of a \f{problem} is either a TPTP problem name, for example
\f{'LCL038-1'}, or indicates a file or other document in TPTP format. (TODO:
maybe use something like \f{'foo:path/to/myfile.p'} for these other files and
documents.)

The following properties are supported for entities of type \f{problem}.

\propsig{goal}{\lit{FormulaTerm}}

\propdesc{The proof goal.}

\propsig{axiom(\var{NatNum})}{\lit{FormulaTerm}}

\propdesc{Axiom number \var{NatNum} of the problem, where \var{NatNum} is a
  natural number, starting from $1$.}

\propsig{number\_of\_axioms}{\lit{NatNum}}

\propdesc{Number of axioms of the problem.}

% \subsubsection*{Not (yet) supported}
% 
% \propsig{axiomset\_identifier}{\f{problem}} 
% 
% \propdesc{Some canonically chosen problem with the same axioms.}

%%%%%%%%%%%%%%%%%%%%%%%%%%%%%%%%%%%%%%%%%%%%%%%%%%%%%%%%%%%%%%%%%%%%%%%%%%%%%%
\subsection{Type \f{proof}}                       

\propsig{problem}{\f{problem}}

\propdesc{The proven problem. Specified as the atom that identifies the
  problem.}

\propsig{source}{\lit{Term}}

\propdesc{Has no internal meaning. Intended to organize fact bases with sets of
  proofs. Can be supplied with some value that indicates the source of the
  proof or its membership in some set of proofs.}


\propsig{meta\_info}{\lit{KeyValueList}}

\propdesc{Has no internal meaning. Used to represent meta information about
  the proof such as how and when it was obtained, with system specific keys.}

\propsig{dcterm}{\lit{DCTerm}}

\propdesc{The proof's D-term (i.e., the structure of proof) as ground
  \lit{DCTerm}.}

\propsig{d\_csize}{\lit{NatNum}}\\
\propsig{d\_tsize}{\lit{NatNum}}\\
\propsig{d\_height}{\lit{NatNum}}

\propdesc{Compacted size, tree size and height of the proof's D-term.}

\subsection{Type \f{lemma}}                       

We consider a lemma in two respects: Its formula and the proof structure
(D-term) that proves the formula from the problem's axioms. The formula is a
logic atom if and only if the D-term is ground. It is a definite clause with
$n$ body atoms if and only if the D-term contains $n$ distinct variables.

For lemmas whose D-term $d_L$ has variables we use different ways to estimate
the size reducing effect of the lemma on a given proof D-term $d_P$, based on
different ways of substituting subterms matching $d_L$ in $d_P$:
\begin{itemize}

\item \textsc{Innermost}: In a preorder traversal replace all instances of
  $d_L$ that have no instance of $d_L$ as subterm with the tuple of the
  variable values according to the match. (If a term is replaced at its visit,
  neither the replaced nor the replacing subterms are traversed.) Instances
  are replaced as they appear in left-to-right depth-first order. The tuple
  constructor is a special symbol that is not counted for size measuring.

\item \textsc{Innermost Exhaustive}: This is \textsc{Innermost} applied
  repeatedly until no more replacements can be performed.
  
\item \textsc{Outermost}: In a preorder traversal replace all instances of
  $d_L$ with the tuple of the variable values according to the match. (If a
  term is replaced at its visit, neither the replaced nor the replacing
  subtrees are traversed.) The tuple constructor is a special symbol that is
  not counted for size measuring.
\end{itemize}

\propsig{problem}{\f{problem}}

\propdesc{The problem for which the lemma is intended. The lemma may be
  extracted from a proof of the problem or generated in some other way. In any
  case, it is assumed that it follows from the problem's axioms.}

\propsig{lf\_proof}{\f{proof}}

\propdesc{The proof associated with the lemma. Specified as the atom that
  identifies the proof. The lemma is not necessarily contained in the proof,
  but can also, e.g., be a negative lemma that relates to the proof. See also
  \f{lfp\_containing\_proof}.}

\propsig{lf\_is\_in\_proof}{NatNum}

\propdesc{1 iff the lemma is contained in some proof. Then also the properties of
  Sect.~\ref{sec-lfp-features} are defined. 0 otherwise.}
  
\propsig{formula}{\f{lemma}(\var{Head},\var{Body})}

\propdesc{The formula of the lemma. \var{Head} is a \lit{FormulaTerm} and
  \var{Body} a \lit{FormulaTermList}. If \var{Body} is empty, the lemma is a
  t-lemma, otherwise it is an s-lemma. \f{lemma}(\var{Head},\var{Body}) may be
  understood as transformation rule of subgoal \var{Head} to subgoals
  \var{Body}, modulo unification. Another way to understand it is as a
  definite clause expressing that \var{Body} implies \var{Head}.}

\propsig{dcterm}{\lit{DCTerm}}

\propdesc{The D-term (proof structure) of the lemma as \lit{DCTerm}. After
  conversion to a \lit{DTerm}, it may be nonground, where the number of
  variables is the same as the length of \var{Body} in the value of
  \f{formula}. Each variable may only have a single occurrence in the
  \lit{DTerm}. (TODO: clarify the purpose of the restriction.)}

\propsig{method}{\lit{Term}}

\propdesc{A term that indicates the method by which the lemma was obtained.
  Currently implemented methods:

  \medskip
  \f{subtree}

  All subtrees (t-lemmas) of the proof's D-term with exception of axioms
  (leaves) and the proof's D-term itself.

  \medskip
  \f{slemma\_nonpure}(\var{Limit},\var{K})\\
  \f{slemma\_pure}(\var{Limit},\var{K})


  \var{Limit} is a \lit{NatNum} or $-1$ or a term \f{h(\lit{NatNum})}. \var{K}
  is a \lit{NatNum}. For all subtrees~$d'$ of the proof's D-term consider all
  trees~$d''$ obtained from~$d'$ by replacing subtrees of~$d'$ with open nodes
  (variables).

  If \lit{Limit} is a \lit{NatNum}, then restrict the subtrees $d''$ to those
  whose tree size does not exceed \var{Limit}. If \lit{Limit} is of the form
  \f{h(\lit{N})}, then restrict the subtrees $d''$ to those whose height does
  not exceed \var{N}. If \lit{Limit} is -1, do not apply such a restriction.
  
  With method \f{slemma\_nonpure}, replace there inner nodes in all possible
  ways, each way leading to a distinct s-lemma. With method \f{slemma\_pure},
  replace all nodes, not just inner nodes, in all possible ways. From the
  resulting s-lemmas remove some trivial ones: single-node trees representing
  an axiom or a variable and remove variants of the D-term \f{d(\_,\_)}. Also
  remove lemmas with only a single match in the proof's D-term. Order the
  s-lemmas by the number of their matches in the proof's D-term, largest
  first, and take the first \var{K} ones as s-lemmas (or all remaining
  s-lemmas if their number is less than \var{K}). So far, these methods do not
  support determining the \f{lf\_d\_incoming} feature property.

   \medskip
   \f{treerepair}
   
   T- and s-lemmas in the combinatory compression
   of the proof's D-term obtained via the TreeRePair tool. Supports the
   \f{lf\_d\_incoming} feature property.
}% propdesc

\subsubsection{Features of the Lemma's D-term}

\propsig{lf\_d\_csize}{\lit{NatNum}}\\
\propsig{lf\_d\_tsize}{\lit{NatNum}}\\
\propsig{lf\_d\_height}{\lit{NatNum}} \propdesc{Compacted size, tree size and
  height of the lemma's D-term.}

\propsig{lf\_d\_grd\_csize}{\lit{NatNum}} \propdesc{Compacted size of the
  lemma's D-term after replacing all variables with 0.}

\propsig{lf\_d\_major\_minor\_relation}{\lit{NatNum}}

\propdesc{Describes the structural relationship of the subproofs of the major
  and minor premise of the lemma. Can have the following values: 0 identical
  or D-term is atomic; 1 is a strict superterm; 2 is a strict subterm; 3 none
  of these relationships; 4 for nonground D-terms.}

\propsig{lf\_d\_number\_of\_terminals}{\lit{NatNum}}

\propdesc{Number of subterms in the lemma's D-term which are of the form
  \f{d($d_1$,$d_2$)} where neither of $d_1,d_2$ is a compound term.}

\subsubsection{Context Dependent Features of the Lemma's D-Term}
\label{sec-lfp-features}

The features depend on the lemma as embedded in a given proof. They are only
available for training data extracted from proofs.

\propsig{lfp\_containing\_proof}{\f{proof}}

\propdesc{The proof that contains the lemma. Specified as the atom that
  identifies the proof. See also \f{lf\_proof}.}

\propsig{lfp\_d\_occs}{\lit{NatNum}} \propdesc{If the lemma's D-term is ground,
  the number of occurrences of the lemma's D-term in the proof's D-term. If
  the lemma's D-term is non-ground, the number of subterm occurrences of the
  proof's D-term that are instances of the lemma's D-term (note that these
  subterm occurrences may overlap).}

\propsig{lfp\_d\_incoming}{\lit{NatNum}} \propdesc{The number of incoming edges
  of the lemma in the minimal DAG representing the proof's D-term. This can
  not be meaningfully determined for all lemma computation methods that yield
  s-lemmas.}

\propsig{lfp\_d\_occs\_innermost\_matches}{\lit{NatNum}} \propdesc{If the lemma's
  D-term is ground, the same as \f{lf\_d\_occs}. If the lemma's D-term has
  variables, the number of subterm occurrences of the proof's D-term that
  would be rewritten by \textsc{Innermost} replacement.}

\propsig{lfp\_d\_occs\_outermost\_matches}{\lit{NatNum}} \propdesc{If the lemma's
  D-term is ground, the same as \f{lf\_d\_occs}. If the lemma's D-term has
  variables, the number of subterm occurrences of the proof's D-term that
  would be rewritten by \textsc{Outermost} replacement.}

\propsig{lfp\_d\_min\_goal\_dist}{\lit{NatNum}} \propdesc{Number of edges in
  the proof's D-term of the shortest downward path from the root to a subtree
  that is an instance of the lemma's D-term.}

\subsubsection{Special Features of the Lemma's Formula Components}
  
\propsig{lf\_b\_length}{\lit{NatNum}}
\propdesc{Length of the \var{Body} component of the lemma's formula.}

\propsig{lf\_hb\_distinct\_hb\_shared\_vars}{\lit{NatNum}}
  \propdesc{Number of distinct
  variables that occur in the \var{Head} as well as in the \var{Body}
  component of the lemma's formula.}

\propsig{lf\_hb\_distinct\_h\_only\_vars}{\lit{NatNum}}
\propdesc{Number of distinct variables that occur in \var{Head} as but not
  the \var{Body} component of the lemma's formula.}

\propsig{lf\_hb\_distinct\_b\_only\_vars}{\lit{NatNum}}
\propdesc{Number of distinct variables that occur in \var{Body} as but not
  the \var{Head} component of the lemma's formula.}

\propsig{lf\_hb\_singletons}{\lit{NatNum}}
\propdesc{Number of distinct variables with a single occurrence in
  \var{Head} and \var{Body} taken together.}

\propsig{lf\_hb\_double\_negation\_occs}{\lit{NatNum}}
\propdesc{Number of instances of \f{n}(\f{n}(\_)) in \var{Head} and \var{Body}.}

\propsig{lf\_hb\_nongoal\_symbol\_occs}{\lit{NatNum}}
\propdesc{Number of occurrences of symbols (functions, constants) in
  \var{Head} and \var{Body} that do not appear in the problem's goal.
  Context-dependent in that it refers to the problem's goal formula.}

\propsig{lf\_h\_excluded\_goal\_subterms}{\lit{NatNum}}
\propdesc{Number of distinct subterms of the goal formula (after replacing
  constants systematicall by variables) that are not a subterm of \var{Head}
  (modulo the variant relationship). Context-dependent in that it refers to
  the problem's goal formula.}

\propsig{lf\_h\_subterms\_not\_in\_goal}{\lit{NatNum}}
\propdesc{Number of distinct subterms of the head that are not a subterm of
  the goal formula (after replacing constants systematicall by variables,
  modulo the variant relationship). Context-dependent in that it refers to the
  problem's goal formula.}

\propsig{lf\_hb\_compression\_ratio\_raw\_deflate}{\lit{NormalizedValue}}\\
\propsig{lf\_hb\_compression\_ratio\_treerepair}{\lit{NormalizedValue}}\\
\propsig{lf\_hb\_compression\_ratio\_dag}{\lit{NormalizedValue}}
\propdesc{Indicates how much the lemma's formula can be compressed. The value
  is roughly compressed size divided by original tree size. That is, formulas
  with ``much regularity'' such that they can be compressed stronger receive
  smaller values. The different properties realize this in variants for
  different notions and implementations of compression. The \f{raw\_deflate}
  version depends on intrinsics of SWI-Prolog's term representation and
  possibly gives different results for the same formula, depending on how it
  internally shares subterms.}

\propsig{lf\_hb\_organic}{\lit{NatNum}}

\propdesc{Whether the formula is organic. A nonenmpty \var{Body} is translated
  to an implication, e.g., if \var{Body} = [a,b,c], the considered formula is
  i(a,i(b,i(c,\var{Head}))). Determined with Minisat. Values: 0: the formula
  is organic; 1 the formula is not organic but weakly organic; 2 the formula
  is not weakly organic. See also
  \url{http://cs.christophwernhard.com/cdtools/downloads/cdtools/pldoc/organic_cd.html}.}

\propsig{lf\_hb\_name}{\lit{Atom}}

\propdesc{A name of the formula if it is well known under some name. For a
  formula with nonempty body the translation to implication is considered (as
  for \f{lf\_hb\_organic}) and the name is prefixed with \f{meta\_}. If the
  formula is not known under some name, the value is \f{zzz}. See also
  \f{named\_axiom}/2 in
  \url{http://cs.christophwernhard.com/cdtools/downloads/cdtools/pldoc/named_axioms_cd.html}.}

\propsig{lf\_hb\_name\_status}{\lit{NatNum}}

\propdesc{Number indicating whether the formula has a name in the sense of
  \f{lf\_hb\_name}: 0 if it has a name and an empty body, 1 if it has a
  nonempty body and a name (prefixed with \f{meta\_}), 2 otherwise.}

\subsubsection{General Features of the Lemma's Formula Components}

These features are specified below schematically with \var{COMP} for \f{h},
\f{b}, and \f{hb}, referring the the respective features for the \var{Head}
component, the \var{Body} component and both of the components joined
together. The schema parameter \var{ITEM} for \f{var}, \f{const} and \f{fun}
refers to variables, constants (atomic values in Prolog syntax) and
function symbols with arity $\geq 1$, respectively.

\propsig{lf\_\var{COMP}\_csize}{\lit{NatNum}}\\
\propsig{lf\_\var{COMP}\_tsize}{\lit{NatNum}}\\
\propsig{lf\_\var{COMP}\_height}{\lit{NatNum}}

\propdesc{Compacted size, tree size and height, respectively. (We use these
  notions, which are also used for D-terms, here for formula terms.)}

\propsig{lf\_\var{COMP}\_distinct\_vars}{\lit{NatNum}}
\propdesc{Number of distinct variables.}

\propsig{lf\_\var{COMP}\_\var{ITEM}\_occs}{\lit{NatNum}}
\propdesc{Number of occurrences of syntactic objects of kind \var{ITEM}.}

\propsig{lf\_\var{COMP}\_occs\_of\_most\_frequent\_\var{ITEM}}{\lit{NatNum}}
\propdesc{Maximum number of occurrences of a syntactic object of kind \var{ITEM}.}

\subsection{Utility Values}

Various utility values (or experimental candidates for utility values). Values
are between 0 and 1, with smaller values indicating better utility.

\propsig{u\_tsize\_reduction}{\lit{NormalizedValue}}\\
\propsig{u\_height\_reduction}{\lit{NormalizedValue}}\\
\propsig{u\_csize\_reduction}{\lit{NormalizedValue}}\\
\propsig{u\_tsize\_reduction\_subst1}{\lit{NormalizedValue}}\\
\propsig{u\_height\_reduction\_subst1}{\lit{NormalizedValue}}\\
\propsig{u\_csize\_reduction\_subst1}{\lit{NormalizedValue}}\\
 \propsig{u\_tsize\_reduction\_subst2}{\lit{NormalizedValue}}\\
\propsig{u\_height\_reduction\_subst2}{\lit{NormalizedValue}}\\
\propsig{u\_csize\_reduction\_subst2}{\lit{NormalizedValue}}
\propdesc{Indicates the reducing effect of the lemma for the respective
  measure, tree size, compacted size and height. The value is calculated as
  \[\m{min}(\m{size}(d_P'), \m{size}(d_P)) / \m{size}(d_P),\]
  where $\m{size}$ is the respective size measure, $d_P$ is the proof's D-term
  and $d_P'$ is obtained from $d_P$ and the lemma's D-term by
  \textsc{Innermost Exhaustive} replacement (the basic versions),
  \textsc{Innermost} replacement (the \f{\_subst1} versions), or
  \textsc{Outermost} replacement (the \f{\_subst2} versions). The minimum
  operator is present on behalf of compacted size, because the replacement
  operations may increase the compacted size.}

\propsig{u\_occs}{\lit{NormalizedValue}} \propdesc{Indicates the number of
  occurrences of the lemma. More occurrences correspond to better utility. For
  lemmas with variables in the D-term \f{lf\_d\_occs\_innermost\_matches} is
  taken as basis of the calculation.}

\propsig{u\_incoming}{\lit{NormalizedValue}}
\propdesc{Indicates the number of incoming edges in the DAG representation.
  A larger number of edges corresponds to better utility.}

\propsig{u\_close\_to\_goal\_path}{\lit{NormalizedValue}} \propdesc{The value
  of \f{lf\_d\_min\_goal\_dist} normalized in relation to the height of the
  proof's D-term.}

\propsig{u\_close\_to\_axioms\_height}{\lit{NormalizedValue}}\\
\propsig{u\_close\_to\_axioms\_tsize}{\lit{NormalizedValue}}
\propdesc{Indicates how close the lemma is to the axioms, to a finishing of
  the proof. These are just normalized versions of height and tree size of
  the lemma's D-term.}


\subsection{Utility Values (II)}

Various utility values (or experimental candidates for utility values). Values
in different range as the other utility values.

\propsig{u\_reproof}{\lit{Float}}

\propdesc{Indicates performance of reproving the problem with the lemma added
  as a single lemma. Values: 1 best, 0: worst, or worse than without lemma,
  -1: lemma does not appear in the proof at all.}

\section{Conversion Utilities}
\label{sec-aux-utils}  

\var{DCTerms} provide a DAG-compressed representation of \var{DTerms}, which
is in particular if proofs are written to files. The following predicates show
the conversion between \var{DTerms} and \var{DCTerms}.

\begin{Verbatim}[fontsize=\small]
d_to_dc(D, A-B) :-
	term_factorized(D, A, B).

dc_to_d(A-B, D) :-
        ( B = [] ->
          D = A
	; copy_term(A-B, D-B1),
	  map_bind(B1)
	).

map_bind([X=X|Y]) :- map_bind(Y).
map_bind([]).
\end{Verbatim}


\end{document}
